\documentclass[11pt,a4paper]{article}

\usepackage{xltxtra}
\usepackage{fontspec} %Font package
\usepackage{xunicode}
\setmainfont{TeX Gyre Pagella}
\setmonofont{Inconsolata}
\usepackage{listingsutf8}
\usepackage[frenchb]{babel} % Global stuff set to french
\usepackage[margin=2cm]{geometry} % The margin of the page
%\usepackage{amsmath}  % to include math formulas
\usepackage{graphicx} % to include pictures
\usepackage[hidelinks]{hyperref} % To include hyperlinks in a PDF
\usepackage{fancyhdr} % to be able to make the page fancy looking
\usepackage{lastpage} % so latex knows what is the last page...
\usepackage{color} % For text colors
%\usepackage{tabularx}

%% Fancy layout
\pagestyle{fancy}
    \lhead{INFO-F203 - Projet Algorithmique 2}
    \chead{}
    \rhead{Carlos Requena (\emph{410031})}
    \lfoot{}
    \cfoot{}
    \rfoot{Page \thepage\ de \pageref{LastPage}}
\renewcommand{\headrulewidth}{0.4pt}
\renewcommand{\footrulewidth}{0.4pt}

\definecolor{mygreen}{rgb}{0,0.6,0}
\definecolor{mygray}{rgb}{0.41,0.41,0.41}
\definecolor{mymauve}{rgb}{0.85,0,0}
\definecolor{myblue}{rgb}{0, 0.2, 0.9}
\definecolor{mybackground}{RGB}{245, 245, 245}


\lstset{ %
  backgroundcolor=\color{mybackground},   % choose the background color; you must add \usepackage{color} or \usepackage{xcolor}
  basicstyle=\normalsize\ttfamily,        % the size of the fonts that are used for the code
  breakatwhitespace=false,         % sets if automatic breaks should only happen at whitespace
  breaklines=true,                 % sets automatic line breaking
  captionpos=b,                    % sets the caption-position to bottom
  commentstyle=\color{mygreen},    % comment style
  columns=flexible,
  deletekeywords={...},            % if you want to delete keywords from the given language
  escapeinside={\%*}{*)},          % if you want to add LaTeX within your code
  extendedchars=true,              % lets you use non-ASCII characters; for 8-bits encodings only, does not work with UTF-8
  frame=trBL,                    % adds a frame around the code
  keepspaces=true,                 % keeps spaces in text, useful for keeping indentation of code (possibly needs columns=flexible)
  keywordstyle=\color{myblue},       % keyword style
  language=java,                 % the language of the code
  morekeywords={*,...},            % if you want to add more keywords
                                % to the set
  inputencoding=utf8,
  numbers=left,                    % where to put the line-numbers; possible values are (none, left, right)
  numbersep=9pt,                   % how far the line-numbers are from the code
  numberstyle=\footnotesize\color{mygray}, % the style that is used for the line-numbers
  rulecolor=\color{black},         % if not set, the frame-color may be changed on line-breaks within not-black text (e.g. comments (green here))
  showspaces=false,                % show spaces everywhere adding particular underscores; it overrides 'showstringspaces'
  showstringspaces=true,          % underline spaces within strings only
  showtabs=false,                  % show tabs within strings adding particular underscores
  stepnumber=2,                    % the step between two line-numbers. If it's 1, each line will be numbered
  stringstyle=\color{mymauve},     % string literal style
  title=\lstname                   % show the filename of files included with \lstinputlisting; also try caption instead of title
}


%%% --- %%% --- DOCUMENT START --- %%% --- %%%
\begin{document}
\pagestyle{fancy}


\section{Introduction}

Le but de ce projet est de produire un algorithme efficace pour
résoudre le casse-tête ``Parking Escape'' également connu sous le nom
``Rush Hour''. Le jeu simule une congestion automobile dans un parking
et l'objectif est de faire sortir une certaine voiture (Goal/rouge)
alignée avec la sortie. Les autres vehicules bloquent la sortie et ne
peuvent se déplacer que le long de ses axes (horizontal ou vertical).

\section{Solution proposée - Algorithme}

Comme conseillé dans l'énoncé, on considère chaque état possible du
parking comme un sommet dans un graphe, et chaque mouvement (déplacer
une voiture de une position) un arc. Le problème devient alors très
simple et il suffit de réaliser un BFS (Breadth First Search), qui
visite (et construit) les noeuds en largeur. L'algorithme peut mettre
fin a son exécution dès qu'une solution a été trouvée, puisque on est
sûr d'avoir trouvé la plus courte et il ne faut pas verifier des
autres solutions.

\medbreak

Plus spécifiquement, on regarde tous les mouvements possibles dans un
sommet/parking de départ \emph{s}. Pour chaque mouvement, un autre
Parking est créé, et il n'est gardé que s'il n'a pas été visité
avant. Aprés avoir construit tous les états possibles avec ces
mouvements on passe au premier et recommence, en regardant tous le
mouvements possibles pour ce parking.

\medbreak

Finalement, si une solution a été trouvée, les étapes sont données
dans l'ordre avec une méthode recursive, qui remonte jusqu'au sommet
de départ grace aux liens entre les arcs et les sommets.

\medbreak

Une autre approche serait de construire un graphe non orienté avec
tous les cas possibles, les relier de la même façon (avec des
arcs/mouvements - on aura des cycles) et utiliser l'algorithme de
Dijkstra pour trouver le chemin le plus court entre un sommet initial
et un sommet marqué comme solution. Dans ce cas, puisque tous les arcs
ont le même poids, cette approche n'est pas choisie.

\subsection{Complexité}

Dans le pire des cas, cet algorithme a une complexité de \emph{O(|S| +
  |A|)}, si \emph{S} est le nombre de sommets et \emph{A} est le
nombre d'arcs, car tous les sommets et tous les arcs sont
visités. L'espace de travail est le même.

\medbreak

Dans notre cas, on connaît pas tous les sommets, donc il est plus
adéquat d'exprimmer la complexité comme: \emph{O(b$^{m})$} où \emph{b}
est le nombre maximum d'arcs sortants d'un sommet (aussi nombre de
sommets ``fils'') et \emph{m} la hauteur maximale.



\subsection{Classes implementées}

\begin{itemize}
\item \texttt{Escaper}: Classe principale, qui contient l'algorithme
  de visite et instancie la class \texttt{Parking} au fur et à
  mesure. Cette classe s'occupe aussi du parsing du fichier et de
  communiquer la solution finale.
\item \texttt{Parking}: État du parking dans un moment
  donné. Plusieurs objets de cette classe forment le graphe.
\item \texttt{Car}: Cette classe représente une voiture (ses
  coordonnées, ID, voiture Goal ou pas, orientation, etc).
\item \texttt{Move}: Contient toute l'information pour savoir quel a
  été le mouvement realisé et quel était l'état du parking juste
  avant. Permet de revenir en arrière facilement. Ces mouvements sont
  les arcs du graphe.
\end{itemize}



\subsection{Parsing fichier input}

Le fichier input passé comme argument au programme est ``parsé'' avec
deux méthodes:

\begin{enumerate}
\item \texttt{parseFile()}: Utilise la classe \texttt{Scanner} et
  \texttt{FileReader} pour l'IO et se sert de la librairie
  \texttt{java.util.regex} pour établir un modèle de coordonnées et
  faire du ``Pattern Matching''.
\item \texttt{extractCar()}: Renvoie un objet de type \texttt{Car}
  avec les bonnes coordonnées.
\end{enumerate}

\subsection{Contraintes}

\begin{itemize}
\item Toutes les voitures doivent être de taille 2.
\item Les fichier input ne doit pas comporter d'erreur et les
  coordonnées doivent être au format \texttt{[(x1,y1), (x2,y2)]}.
\item La voiture Goal doit se trouver en première position dans le
  fichier input.
\end{itemize}

\subsection{Structures de données}

Des listes de type \texttt{List<TYPE>, ArrayList()} sont utilisés dans
la plus part des cas. Une file \texttt{Queue<TYPE>, LinkedList()} est
aussi utilisé pour stocker les sommets a visiter dans l'algorithme de
visite. En effet, celui ci se base sur cette file pour accéder aux
sommets en largeur (``breadth first'').

\section{Améliorations}

\begin{itemize}
\item Même si le paradigme OOP a été suivi, beaucoup de proprietés des
  objets restent publiques, pour faciliter l'accés aux autres classes,
  et ne pas gonfler les classes avec des accesseurs. Néanmoins, une
  meilleure encapsulation peut être achevée.
\item On pourrait considérer le déplacement de plusieurs positions
  d'une voiture un mouvement. Cela reduit le nombre des ``mouvements
  effectifs'', mais la solution reste la même.
\item Pour reduire le temps d'exécution (proportionel au nombre de
  voitures cette fois), la vérification d'emplacements libres pourrait
  se faire quand on est sûr que ce parking est nouveau et n'a jamais
  été visité. Dans l'algorithme proposé, tout est fait au même temps
  dans la même méthode (\texttt{updateGrid()}).
\item Une nouvelle class \texttt{Truck} pourrait être créé pour gérer
  les cas des vehicules qui occupent 3 positions.

\end{itemize}


\newpage

\section{Code listing}

\lstinputlisting[language=java]{Escaper.java}
\newpage
\lstinputlisting[language=java]{Parking.java}
\newpage
\lstinputlisting[language=java]{Car.java}
\newpage
\lstinputlisting[language=java]{Move.java}

\end{document}
