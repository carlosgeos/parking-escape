\documentclass[11pt,a4paper]{article}

\usepackage[T1]{fontenc}
\usepackage[utf8]{inputenc}
\usepackage[frenchb]{babel} % Global stuff set to french
\usepackage[margin=2cm]{geometry} % The margin of the page
%\usepackage{amsmath}  % to include math formulas
\usepackage{graphicx} % to include pictures
\usepackage[hidelinks]{hyperref} % To include hyperlinks in a PDF
\usepackage{fancyhdr} % to be able to make the page fancy looking
\usepackage{lastpage} % so latex knows what is the last page...
\usepackage{color} % For text colors
\usepackage{palatino} % Change font
%\usepackage{tabularx}
\usepackage{listings}

%% Fancy layout
\pagestyle{fancy}
    \lhead{INFO-F203 - Projet Algorithmique 2}
    \chead{}
    \rhead{Carlos Requena (\emph{410031})}
    \lfoot{}
    \cfoot{}
    \rfoot{Page \thepage\ de \pageref{LastPage}}
\renewcommand{\headrulewidth}{0.4pt}
\renewcommand{\footrulewidth}{0.4pt}

\definecolor{mygreen}{rgb}{0,0.6,0}
\definecolor{mygray}{rgb}{0.41,0.41,0.41}
\definecolor{mymauve}{rgb}{0.85,0,0}
\definecolor{myblue}{rgb}{0, 0.2, 0.9}
\definecolor{mybackground}{RGB}{245, 245, 245}


\lstset{ %
  backgroundcolor=\color{mybackground},   % choose the background color; you must add \usepackage{color} or \usepackage{xcolor}
  basicstyle=\normalsize\ttfamily,        % the size of the fonts that are used for the code
  breakatwhitespace=false,         % sets if automatic breaks should only happen at whitespace
  breaklines=true,                 % sets automatic line breaking
  captionpos=b,                    % sets the caption-position to bottom
  commentstyle=\color{mygreen},    % comment style
  columns=flexible,
  deletekeywords={...},            % if you want to delete keywords from the given language
  escapeinside={\%*}{*)},          % if you want to add LaTeX within your code
  extendedchars=true,              % lets you use non-ASCII characters; for 8-bits encodings only, does not work with UTF-8
  frame=trBL,                    % adds a frame around the code
  keepspaces=true,                 % keeps spaces in text, useful for keeping indentation of code (possibly needs columns=flexible)
  keywordstyle=\color{myblue},       % keyword style
  language=c,                 % the language of the code
  morekeywords={*,...},            % if you want to add more keywords to the set
  numbers=left,                    % where to put the line-numbers; possible values are (none, left, right)
  numbersep=9pt,                   % how far the line-numbers are from the code
  numberstyle=\footnotesize\color{mygray}, % the style that is used for the line-numbers
  rulecolor=\color{black},         % if not set, the frame-color may be changed on line-breaks within not-black text (e.g. comments (green here))
  %showspaces=false,                % show spaces everywhere adding particular underscores; it overrides 'showstringspaces'
  %showstringspaces=false,          % underline spaces within strings only
  showtabs=false,                  % show tabs within strings adding particular underscores
  stepnumber=2,                    % the step between two line-numbers. If it's 1, each line will be numbered
  stringstyle=\color{mymauve},     % string literal style
  title=\lstname                   % show the filename of files included with \lstinputlisting; also try caption instead of title
}


%%% --- %%% --- DOCUMENT START --- %%% --- %%%
\begin{document}
\pagestyle{fancy}


\section{Introduction}

Le but de ce projet est de produire un algorithme efficace pour
résoudre le problème dessss

\section{Solution proposé - Algorithme}

Comme conseillé dans l'énoncé, on considere chaque état possible du
parking comme un sommet dans un graphe, et chaque mouvement (déplacer
une voiture de une position) un arc. Le problème devient alors très
simple et il suffit de réaliser un BFS (Breadth First Search), qui
visite les noeuds en largeur.

Plus spécifiquement,

Dans ce cas, puisque tous les arcs ont le même poids,


\subsection{Parsing fichier input}

Le fichier input passé comme argument au programme est ``parsé'' avec
deux méthodes:

\begin{enumerate}
\item \texttt{parseFile()}: Utilise la classe \texttt{Scanner} et
  \texttt{FileReader} pour l'IO et se sert de la librairie
  \texttt{java.util.regex} pour établir un modèle de coordonnées et
  faire du ``Pattern Matching''.
\item \texttt{extractCar()}: Renvoie un objet de type \texttt{Car}
  avec les bonnes coordonnées.
\end{enumerate}

\subsection{Contraintes}

\begin{itemize}
\item Toutes les voitures doivent être de taille 2.
\item Les fichier input ne comporte pas d'erreur.
\end{itemize}

\subsection{Structures de données}

List...

\section{Améliorations}

Même si le paradigme OOP a été suivi, beaucoup de proprietés des
objets restent publiques, pour faciliter l'accés aux autres classes,
et ne pas gonfler les classes avec des accesseurs. Néanmoins, une
meilleure encapsulation peut être achevée.

Un autre algorithme heuristique..


\newpage

\section{Code listing}

\lstinputlisting[language=java]{Escaper.java}
\newpage
\lstinputlisting[language=java]{Parking.java}
\newpage
\lstinputlisting[language=java]{Car.java}
\newpage
\lstinputlisting[language=java]{Move.java}
\newpage
\lstinputlisting[]{input.txt}
\newpage
\lstinputlisting[]{output.txt}

\end{document}
